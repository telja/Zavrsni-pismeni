\section{Zaključak}
Cilj svake igre je da bude korisniku što jednostavnija za upotrebu i naravno da bude zanimljiva. Svaki programer može napraviti igru, ali pitanje je koliko će ta igra privlačiti korisnika. Da bi igra bila pristupačnija korisniku trebala bi biti lijepo dizajnirana jer većina korisnika koja želi igrati neku igru će prvo pogledati dizajn igre. Možemo reći da će dizajn igre privuči korisnika a fukncionalnosti igre će pridobiti korisnika. Učenje svake tehnologije je u početku teško ali ukoliko je programer željan stvaranje nečeg novog i ima strast prema igrama unity je pravo rješenje za nega. Na internetu se može naći puno primjera i objašnjenja kako koristiti unity. Tjeko izrade ovog projekta sam se upoznao sa unity okolinom. Trebalo mi je nekoliko vremena da savladam glavne komponente unity-a, a nakon što sam to prošao samo sam trebao upotrebiti vlastitu maštu kako bih igru napravio što pristupačnijom korisniku. Kod izrade igara za mobilne uređaja se javlja problem kontrole nekog objekta. Objekte možemo kontrolirati preko dugmadi, dodira na ekran i preko akcelometra. Ja sam se odlučio za akcelometar zato što time ostavljam korisniku čist pogled na igru, ali sve ovisi o kompleksnosti igre koje ce autor kontrolu odabrati. 
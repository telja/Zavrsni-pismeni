\section{Objekti}



\subsection{Player}
Ovo je glavni objekt u igri. Napravljen je od više objekata u blenderu. S ovim objektom upravlja klasa "packmanController". Sadrži sljedeće komponente:
\begin{itemize}
\item Transform --> Zadana komponenta svakog objekta koja određuje poziciju, rotaciju i veličinu objekta.

\item Rigidbody --> Kontrolira poziciju objekta kroz fizičku simulaciju. Ovaj objekt ne koristi fiziku, ali je iskorištena njegova komponenta da se onemogući kretnja objekta po y osi.

\item Sphere collider --> Služi za detekciju kolajdera u obliku kule. Time možemo postaviti veličinu radiusa za kolajder.

\item Audio source --> Pridružen je objektu zbog ispuštanja zvuka kojeg objekt kontrolira.

\item Animator --> Služi za kontrolu animacije objekta. U ovoj igri aniamcija je napravljena u blender alatu. Objekt preko ove komponente upravlja sa svojom animacijom.

\end{itemize}

\begin{center}
	\includegraphics[scale=0.50]{player.png}
	
	Slika 5: Objekt "Player"
\end{center}
\subsection{PickUps}
Ovo je prazan objekt u kojemu su smješteni objekti "PickUp" koje igrač skuplja u igri. S tim objektima upravlja klasa "Rotator". Ovi objekti imaju i tag naziva "pickUp" kako bi se lakše pristupilo svim objektima odjednom. Njihove komponente su:

\begin{itemize}
\item Transform

\item Box Collider --> Služi za detekciju kolajdera u obliku kocke.
Možemo postaviti x, y i z veličinu kolajdera.

\item Mesh renderer --> Pomoću njega dodjeljujemo materijale objektu(boja), možemo definirati veličinu i sjenu. 

\end{itemize}

\subsection{Ghost}
Postoje pet istih objekata "Ghost". Ovaj objekt je napravljen u blender alatu i kasnije se duplicirao u unity-u. Njima je pridružen tag "ghost" i "ghostCollor". Preko taga možemo upravljat s njima iz skripte drugog objekta. Njima upravlja klasa "ghostController" i svi imaju iste komponente:

\begin{itemize}
\item Transform

\item Mesh renderer

\item Capsule collider --> Služi za detekciju kolajdera u obliku kapsule. Možemo postaviti visinu i radijus kolajdera. Kolajder sadrži svojstvo "isTrigger" koje se koristi za "triggering" događaje.


\begin{verbatim}
void OnTriggerEnter(Collider other)
{
other.gameObject.SetActive(false);
}
\end{verbatim}
\begin{center}
	
	Ispis 1: Prikaz funkcije "OnTriggerEnter()"
\end{center}
\item Character controller --> Omogućuje kretanje objekta bez da se koristi Rigidbody i da se sudara sa drugim objektima tj. da ne prolazi kroz njih.

\item Audio source

\item Animator

\item Nav Mesh Agent --> Omogućuje nam da se objekti ne sudaraju međusobno i sa drugim objektima tokom kretnje. Jedna od prednosti ove komponente je da će se objekt uvjek rotirati prema smjeru kretnje. Radius i visina objekta nam služi da bi odredili koliziju sa drugim objektima.

\end{itemize}

\begin{center}
	\includegraphics[scale=0.50]{ghost.png}
	
	Slika 6: Objekt "Ghost"
\end{center}
\subsection{Bonuses}
Imamo četiri različita objekta bonus koji imaju iste komponente ali služe za različite stvari koje se kontroliraju u "packmanController" klasi. Ovim objektima je pridružena skripta Yrotator koja ih rotira po y osi radi boljeg ugođaja tokom igranja igre. Nazivi objekata su: "ghostBusters", "ghostBuster", "lifeBonus" i "timeBonus". Njihove omponente su:

\begin{itemize}
\item Transform

\item Mesh renderer

\item Capsule colider

\end{itemize}

\subsection{Ground}
Ovaj objekt je podloga na kojoj se igra odvija. Komponente podloge su:

\begin{itemize}
\item Transform

\item Mesh renderer

\end{itemize}
\subsection{BackGround}
Ovaj objekt je podloga koja se nalazi ispod glavne podloge na koji je nalijepljena slika zbog boljeg ambijenta tokom igranja.
Ima iste komponente kao objekt "ground".

\subsection{Platno}
Platno je objekt koji u sebi sadržava sve UI elemente. Kada želimo dodati neki tekst ili sliku automatski se kreira platno koje će u sebi sadržavati tu sliku ili tekst. Platno je roditelj svih UI elemenata. Ukoliko smanjujemo ili povečavamo prozor projekta platno će se automatski prilagoditi veličini prozora. Ovaj objekt u sebi sadži pet tekst komponenti i dva botuna. Tekst komponente služe za ispisivanje određenog sadržaja, a botuni za navigaciju kroz scene.

\subsection{Direction light}
Služi nam za postavljanje pozicije svjetla, namještanje sjene, jačina svijetla itd. Te opcije se nalaze u njenoj zadanoj komponenti "Light". Kao i svi drugi objekti sadrži transform komponentu koja određuje iz kojeg će kuta dolaziti sjena. 

\subsection{Main Camera}
Glavna kamera u igri koja prati igrača tokom njegove kretnje. Kretnju tokom igre mu omogućuje klasa "cameraController" koja je priključena na ovaj objekt. Objektu je još dodjeljena i "Main Menu" klasa koja navigira između scena. "Main Camera" ima i svoju zadanu komponentu "Camera" gdje možemo postaviti boju neba, širinu pogleda kamere, pozadinu itd.




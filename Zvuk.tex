\section{Zvuk}

Da bi igra bila korisniku što poželjnija potrebno je u nju postaviti zvuk. Zvuk daje korisniku osjećaj da stvarno upravlja sa nekim objektom. Glavna komponenta zvuka u Unity-u je "Audio Source". On reproducira zvuk u sceni preko "Audio Clip" svojstva. U tablici su prikazani formati zvuka koje podržava unity.  


\begin{center}
	\includegraphics[scale=0.75]{audioFormat.png}
	
	
	Slika 5: Formati zvuka koje podržava Unity
\end{center}


\subsection{Postavljanje zvuka}
Postavljanje zvuka u unity-u je vrlo jednostavno. Da bismo stvorili zvuk koji če se koristiti tokom igre potrebno je taj zvuk prvo prebaciti u unity projekt. Kada je zvuk u projektu dodjeli se objektu sa kojim ima komponentu "Audio Source". "Audio Source" komponenta sadrži mnogo svojstva. Jedno od svojstva je "Play on Awake" koje se koristi u "Main Menu" sceni. Ovo svojstvo pokreće željeni zvuk kada se igra pokrene. Unity svaku datoteku zvuka gleda kao "Audio Clip".

\subsection{Audio Clip}
"Audio Source" komponenta je beskorisna bez svojstva "Audio Clip". "Audio Source" je zapravo kontrola koja upravlja sa svojstvom "Audio Clip". Ako koristimo samo jedan zvuk dovoljno ga je samo ubaciti u objekt, a unity će sam napraviti komponentu i ubaciti taj zvuk u nju.

Primjer ispuštanja zvuka:

AudioSource audio;

public AudioClip death;

audio = GetComponent<AudioSource>();

audio.PlayOneShot(death);



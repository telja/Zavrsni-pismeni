\section{Uvod}
Tema završnog rada je izrada igre na unity platformi. Unity platforma je za mnoge najbolja platforma za izradu igara jer je jednostavna za shvatiti te uz malo znanja c sharp ili java script programskog jezika svako može kreirati željenu igru. U svom radu koristio sam c sharp programski jezik jer pruža više mogućnosti za rad. Kolegiji koji su mi pomogli za izradu igre su: programske metode i apstrakcije, programiranje u c sharp-u, napredno windows programiranje i strukture podataka i algoritmi.
Kroz izradu igre sam se postepeno upoznavao sa Unity platformom koja je iznimno korisna i jednstavna za izradu igara na željenim konzolama.
Za ovakav rad me motivirala želja za izradom igara te trenutno tržište prodaje igara koje raste iz dana u dan.
U drugom poglavlju su navedene i kratko opisane tehnologije koje su se koristile pri izradi igre. To su Unity i Blender.
Treće poglavlje opsežno opisuje unity platformu kako bi se predočila kompleksnost unity-a. Osnove koje se trebaju znati za izradu igre se nalaze u ovom poglavlju.
U četvrtom poglavlju su opisane scene u igri. Ova igra sastoji se od dvije scene "MainMenu" i "Game". Svaka scena za sebe ima svoje objekte i skripte.
Peto poglavlje prikazuje i opisuje sve objekte koji su korišteni u igri. Bez objekata igra nije igriva te su sve komponente svakog objekta objašnjene u petom poglavlju.
Šesto poglavlje opisuje klase. Klase upravljaju objektima i u ovom poglavlju se vide sve funkcionalnosti svakog objekta.
Sedmo poglavlje su animacije. U ovom poglavlju je opisana izrada animacija i objašnjenje zašto su animacije važne.
Osmo poglavlje objašnjava važnost zvuka i njegovu upotrebu a u devetom poglavlju se opisuju mogučnosti izrade igre na različitim platformama.

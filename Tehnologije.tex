\section{Korištene tehnologije}

Za izradu ovog rada koristili su se Unity i Blender alati. 
\subsection{Unity}

Unity je najbrže rastući game engine za izradu 2D i 3D igara na svjetu. Jednostavan je za korištenje, može se raditi u timu, podržava bitne platforme kao što su: PC, Mac, Linux, Android, iOS, xBox, Play Station, itd.
Sve što od vas zahtjeva je znjanje c sharp programskog jezika. Može se koristiti i java script programski jezik, ali puno je jednostavnije zbog pronalazaka pogreški tokom pisanja skripte, koristiti c sharp programski jezik. Skripte se mogu pisati u bilo kojem editoru. Ukoliko korisnik želi napraviti unikantnu igru bilo bi poželjno da zna samostalno izrađivati 3D ili 2D objekte. Samostalno izrađivanje objekata nije nužno jer danas postoji mnogo gotovih objekata koje korisnik može besplatno preuzeti i koristiti se s njima. 
Unity se može besplatno koristiti te se može na njemu zarađivati. Ukoliko autor neke igre zaradi preko 100.000 dolara tada mora platiti profesinalnu verziju Unity-a.

 
\subsection{Blender}

Blender je besplatan alat za za izradu 3D računalne grafike koji se koristi za izradu animiranih filmova, vizualnih efekata, modela za 3D printere, itd.
Blender je razvila softverska kuća "Not a Number Technologies". Radna površina blendera je podjeljena na tri djela: glavni meni na vrhu, središnji 3D prostor za modeliranje i prostor sa dugmićima i opcijama na dnu prozora. Nakon što se kreira model potrebno je kreirati datoteku sa 3ds ekstenzijom te je takvu ubaciti u Unity alat.